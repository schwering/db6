\documentclass[mathserif]{beamer}
\usepackage[orientation=landscape,
            size=custom, width=48, height=27,
            scale=0.6, debug]{beamerposter}
\usepackage{amsmath, amsthm, amssymb}
\usepackage[latin1]{inputenc}
\usepackage[english]{babel}
\usepackage{beamerthemesplit}
\setbeamertemplate{navigation symbols}{}

\usetheme{Frankfurt}

\newcommand \BTree { B\textsuperscript{+}-Tree }
\title[Dingsbums \BTree]{\BTree with variable-length entries}
\author{Christoph Schwering}
\institute{RWTH Aachen}
\date{\today}

\begin{document}
\section{Dingsbums \BTree}
\subsection{\BTree with variable-length entries}
\begin{frame}[t]{}
\begin{columns}[t]

\begin{column}{.27\linewidth}

    \begin{block}{Subject}
    \begin{itemize}
    \item \BTree with bounded-length entries
    \item each node is stored in one disk block
    \item at least $\tfrac{3}{8}$ of each block is used
    \item only sizes of keys \& values vary!
    \end{itemize}
    \end{block}

    \begin{block}{Definitions}
    \begin{itemize}
    \item $N$ is a node $N$ with sorted entries $e \in N$
    \item following accessor functions: $parent(N)$, $left(N)$, $right(N)$,
        $key(e)$, $value(e)$ (leafs only), $count(e)$ \& $child(e)$ (inner nodes
        only)
    \item $size(N) = \text{fix overhead} + \sum_{e \in N} size(e)$\\
        $size(e) = \text{fix overhead} + size(key(e)) + size(value(e))$\\
    \item $B = \text{disk block size} - \text{fix node overhead}$
    \end{itemize}
    \end{block}

    \begin{alertblock}{Bounds}
    For root $R$, node $N \neq R$ and any entry $e$, these invariants must hold:
    \[\begin{array}{rclr}
    & size(e) & \leq \lfloor \tfrac{1}{4} B \rfloor\\
    \lfloor \tfrac{3}{8} B \rfloor \leq & size(N) & \leq B
        & \hspace{2cm}\text{(*)}\\
    0 \leq & size(R) & \leq B
    \end{array}\]
    \end{alertblock}

    \begin{block}{Overflow-Theorem}
    A node with entries \mbox{$N \cup \{ e \}$} with
    \[ size(N) \leq B \text{ but } size(N \cup \{e\}) > B \]
    can be split into two nodes such that both fulfill (*).
    \end{block}

    \begin{block}{Underflow-Theorem}
    When an entry $e$ is removed from a non-root-node with entry set $L$ such
    that $size(L) \geq \lfloor \tfrac{3}{8} B \rfloor$ but
    $size(L \setminus \{e\}) < \lfloor \tfrac{3}{8} B \rfloor$, either
    entries from a neighbor can be moved or $L$ can be merged with a neighbor.
    \end{block}

\end{column}

\begin{column}{.27\linewidth}
    \vfill
    \Large
    \begin{itemize}
    \item Someone has an idea what to put here?
        \\
    \item That's my name,\\ask me again and I'll tell you the same.
        \\
    \item San Diego here I come,\\Melmac's where I started from.
        \\
    \item Kate, you've got to hear this.
    \end{itemize}
\end{column}

\begin{column}{.27\linewidth}

    \footnotesize

    \begin{block}{Proof of Overflow-Theorem}
    For \mbox{$N \cup \{e\} = \{ e_1, \ldots, e_n \}$}, set
    \mbox{$L := \{ e_1, \ldots, e_i \}$} such that
    \[ size(L) \geq \lfloor \tfrac{3}{8} B \rfloor \text{ but }
            size(L \setminus \{e_i\}) < \lfloor \tfrac{3}{8} B \rfloor \]
    and \mbox{$R := (N \cup \{e\}) \setminus L$}.
    Then $L$ and $R$ fulfill (*):
    \begin{align*}
    size(L) &\stackrel{\mathrm{def}}{\geq} \lfloor \tfrac{3}{8} B \rfloor\\
    size(L) &< \lfloor \tfrac{3}{8} B \rfloor
                + \underbrace{size(e_i)}_{\leq \lfloor \frac{1}{4} B \rfloor}
            \leq \lfloor \tfrac{5}{8} B \rfloor\\
    size(R) &= \overbrace{size(N \cup \{e\})}^{> B}
                    - \overbrace{size(L)}^{< \lfloor \frac{5}{8} B \rfloor}
            > \lceil \tfrac{3}{8} B \rceil\\
    size(R) &= \underbrace{size(N \cup \{e\})}_{
                            \leq B + \lfloor \frac{1}{4} B \rfloor}
                    - \underbrace{size(L)}_{\geq \lfloor \frac{3}{8} B \rfloor}
            \leq \lceil \tfrac{7}{8} B \rceil
    \end{align*}
    \end{block}

    \begin{block}{Proof of Underflow-Theorem}
    Let $\delta := \lfloor \tfrac{3}{8} B \rfloor - size(L \setminus \{e\})$
    be the space in \mbox{$L \setminus \{e\}$} that needs to be filled to 
    fulfill (*).
    Without loss of generality, we assume that the right neighbor with entry
    set $R$ of $L$ exists; otherwise all left/right and min/max
    relations turn around.

    \begin{enumerate}
    \item
    If \mbox{$size(R) \leq \lceil \frac{5}{8} B \rceil + \delta$}, the nodes
    can be merged to a single node \mbox{$E := (L \setminus \{e\}) \cup R$}
    that fulfills (*):
    \begin{align*}
    size(E)
            &= \overbrace{size(L \setminus \{e\})}^{
                            \geq \lfloor \frac{3}{8} B \rfloor 
                                    - \lfloor \frac{1}{4} B \rfloor}
                    + \overbrace{size(R)}^{
                            \geq \lfloor \frac{3}{8} B \rfloor}
            \geq \lfloor \tfrac{1}{2} B \rfloor\\
    size(E)
            &= \underbrace{size(L \setminus \{e\})}_{
                            \leq \lfloor \frac{3}{8} B \rfloor - \delta}
                    + \underbrace{size(R)}_{
                            \leq \lceil \frac{5}{8} B \rceil + \delta}
            \leq B
    \end{align*}
    \item
    If \mbox{$size(R) > \lceil \frac{5}{8} B \rceil + \delta$}, a set
    \mbox{$S \subseteq R$} must be moved from $R$ to $L$.
    Set \mbox{$L' := (L \setminus \{e\}) \cup S$} and
    \mbox{$R' := R \setminus S$}.
    $S$ must be chosen so that it holds
    \[ size(L') \geq \lfloor \tfrac{3}{8} B \rfloor \text{ but }
            size(L' \setminus \max S) < \lfloor \tfrac{3}{8} B \rfloor. \]
    Hence, we have
    \[ \delta \leq size(S) < \delta + \lfloor \tfrac{1}{4} B \rfloor. \]
    It follows that $L'$ and $R'$ fulfill (*):
    \begin{align*}
    size(L')
            &\stackrel{\mathrm{def}}{\geq} \lfloor \tfrac{3}{8} B \rfloor\\
    size(L')
            &= \underbrace{size(L \setminus \{e\})}_{
                            = \lfloor \frac{3}{8} B \rfloor - \delta}
                    + \underbrace{size(S)}_{
                            < \delta + \lfloor \frac{1}{4} B \rfloor}
            < \lfloor \tfrac{5}{8} B \rfloor\\
    size(R')
            &= \overbrace{size(R)}^{
                            > \lceil \frac{5}{8} B \rceil + \delta}
                    - \overbrace{size(S)}^{
                            < \delta + \lfloor \frac{1}{4} B \rfloor}
            > \lfloor \tfrac{3}{8} B \rfloor\\
    size(R')
            &< size(R) \leq B
    \end{align*}
    \end{enumerate}
    \end{block}

\end{column}


\end{columns}
\end{frame}
\end{document}

