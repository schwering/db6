% vim:foldmethod=marker:tabstop=4:shiftwidth=4
\documentclass[a4paper, 8pt, twoclumn]{scrartcl}
\usepackage[english]{babel}
\usepackage[latin1]{inputenc}
\usepackage{amsthm,amsmath,amssymb}
\usepackage{typearea}
\usepackage{multicol}
\usepackage{natbib}
\usepackage{times}


\setlength{\columnsep}{9mm}
%\setlength{\columnseprule}{.1pt}
\areaset{0.8\paperwidth}{0.8\paperheight}

\theoremstyle{plain}

\theoremstyle{definition}
\newtheorem{thm}{Theorem}[section]
\newtheorem{lem}[thm]{Lemma}
\newtheorem{prop}[thm]{Proposition}
\newtheorem{algo}[thm]{Algorithm}
\newtheorem{proc}[thm]{Procedure}
\newtheorem{defi}[thm]{Definition}
\newtheorem{notation}[thm]{Notation}

\theoremstyle{remark}
\newtheorem{rem}[thm]{Remark}

\newcommand{\BTree}{ B\textsuperscript{+}-Tree }
\newcommand{\BTrees}{ B\textsuperscript{+}-Trees }
%\newcommand{\Beff}{ B_{\text{eff}} }
\newcommand{\Beff}{B_e}
\newcommand{\Lfloor}{\left\lfloor}
\newcommand{\Rfloor}{\right\rfloor}
\newcommand{\Lceil}{\left\lceil}
\newcommand{\Rceil}{\right\rceil}

% don't hyphenate so much - default = 200, max (never hyphenate) = 10,000
%\hyphenpenalty=10000

\title{Dingsbums \BTree}
\date{\today}
\author{Christoph Schwering}

\begin{document}
\bibliographystyle{plain}


\maketitle

\begin{abstract}
This article gives a formal description of the \BTree used in the Dingsbums 6
database system.
A \BTree is an order-preserving index structure.
The B-Tree was first described by Bayer and McCreight \cite{Bayer}.
Several variants of the B-Tree appeared in the aftermath, in particular the
\BTree \cite{Cormen, KnuthBTree}, which differs from the B-Tree in that it
carries the values only in its leaves.
Knuth also mentions a variant that is capable of storing variable-length
keys, but does not describe it further \cite{KnuthBTree}.

The described \BTree supports bounded length keys and values and access by 
the index number of entries.
\end{abstract}


\begin{multicols}{2}


\section{Definition} %%%%%%%%%%%%%%%%%%%%%%%%%%%%%%%%%%%%%%%%%%%%%%%%%%%%%%%%%%

\begin{defi}[Node]
A node \mbox{$N = (M, E)$} of degree \mbox{$d = deg(N)$} consists of some
meta data $M$ and an entry set \mbox{$E = \{ e_1, \ldots, e_d \}$} of
$d$ entries.

The meta data stores the degree $deg(N)$ of the node and the left neighbor,
the right neighbor and the parent, denoted by $left(N)$, $right(N)$, $parent(N)$
respectively. If such a neighbor or parent node does not exist, the
value is $\bot$.

The form of an entry depends on the position of the node.
An entry $e_i$ of a {\em inner node} consists of
\begin{itemize}
\item a key value $key(e_i)$,
\item a count $count(e_i)$, and
\item an address $child(e_i)$.
\end{itemize}
An entry $e_i$ of a {\em leaf node} consists of
\begin{itemize}
\item a key value $key(e_i)$, and
\item a data value $value(e_i)$.
\end{itemize}
For inner nodes, we write \mbox{$e = (k, c, a)$} and for inner nodes 
\mbox{$e = (k, v)$}
if the context permits it.
If the context permits it, we do not distinguish between addresses and nodes,
i.e. \mbox{$parent(child(e_i)) = N$}, for example.
If $N$ is a leaf node, $count(e)$ is defined as $1$.
\end{defi}


\begin{figure*}[tb]
\newdimen \shortlinewidth \shortlinewidth=0.75\linewidth
\centering
\begin{minipage}{\shortlinewidth}
\begin{align}
& size(e) \leq \Lfloor \tfrac{1}{n} \Beff \Rfloor
	&& \text{ for all } e \in N \label{maxsize}\\
& \Lfloor \tfrac{n-2}{2n} \Beff \Rfloor \leq size(N) \leq \Beff
	&& \text{ for all } N \text{ except the root} \label{minsize}\\
& 0 \leq size(R) \leq \Beff
	&& \text{ for the root node } R \label{rootminsize}
\end{align}
\caption{Node filling requirements}
\hspace{2cm}
\hrule
\end{minipage}
\label{fill}
\end{figure*}

\begin{defi}[Tree]
A \BTree is defined over a domain $K$ of keys that is fully ordered by
\mbox{$\leq \ \subseteq K \times K$} and a domain $V$ of values.
We write \mbox{$e_i \leq e_j$} in the following iff
\mbox{$key(e_i) \leq key(e_j)$}.
The tree supports the operations {\em insertion}, {\em deletion} and 
{\em search}. We can define these operations by looking at the states of the
tree:
\begin{itemize}
\item In state $S_0$, the tree is empty and for all $k$,
	\mbox{$search(S_0, k) = \bot$} (i.e. no value $v$ for $k$ can be found)
\item In a state $S_i$, the operation \mbox{$insert(S_i, k, v)$} reaches a
	state $S_{i+1}$ in which \mbox{$search(S_{i+1}, k) = v$},
	if \mbox{$search(S_i, k) = \bot$}. Otherwise, the insertion fails and the
	tree state does not change.
\item In a state $S_i$, the operation \mbox{$delete(S_i, k)$} returns $v$ and
	reaches a state $S_{i+1}$ in which \mbox{$search(S_{i+1}, k) = \bot$}.
\end{itemize}
In all states, the key condition which makes searching efficient must hold
for all nodes \mbox{$N = (M, E)$} in the tree:
\begin{align*}
\forall e \in E: \ & (child(e) = (N' = (M', E')))\\
     \Rightarrow \ & (\forall e' \in E': e' \leq e).
\end{align*}
Furthermore, the counts must be consistent in all states:
\begin{align*}
\forall e \in E: \ & child(e) = (N' = (M', E'))\\
     \Rightarrow \ & count(e) = \sum_{e' \in E'} count(e').
\end{align*}

\noindent {\em Size Limits.}
Normal \BTrees require their nodes to have a certain fixed
minimum and maximum degree, because each node is intended to be stored in one
hard disk block of size $B$ (typically $4096$ bytes).
In contrast, our tree is not intended to support keys of a fixed size
but {\em variable length keys}.
(Note that the sizes of child addresses, count values and data values must be
fixed, though.)

Hence, if $size(k)$ is the size of a certain key $k \in K$, we can uniquely
determine the size of each entry $e$ which will be denoted by $size(e)$.
Then \mbox{$size(E) = \sum_{e \in E} size(e)$} denotes the size of a complete
entries \mbox{$N = (M, E)$}.
Since the size of the meta data is fixed, it is easy to compute 
\mbox{$size(N) = size(M) + size(E)$}.
Furthermore we will write \mbox{$\Beff = B - size(M)$}.
Finally, we require for each node \mbox{$N = (M, E)$} two hold the requirements
depicted in figure~\ref{fill}.
\end{defi}


\begin{rem}[Key size]
Note that req.~\ref{maxsize} refers to a node element $e$ which consists
of can be either \mbox{$e = (k, c, a)$} for a key $k$, a count value $c$ and
a child address $a$ or \mbox{$e = (k, v)$} for a key $k$ and a value $v$.
It follows that the maximum key size is
\[ size(k) \leq \Lfloor \tfrac{1}{n} \Beff \Rfloor
		- \max \{ size(c) + size(a), size(v) \}. \]
This upper bound is statically determined, because the sizes of child addresses,
count values and data values must be fixed.
\end{rem}



\section{Theorems} %%%%%%%%%%%%%%%%%%%%%%%%%%%%%%%%%%%%%%%%%%%%%%%%%%%%%%%%%%%%

\begin{thm}[Node-Overflow]
A node \mbox{$N' = (M, E \cup \{ e \})$} with 
\[ size(E) \leq \Beff \text{ but } size(E \cup \{e\}) > \Beff \]
can be split into two nodes such that both fulfill req.~\ref{minsize}.

\begin{proof}
If \mbox{$E \cup \{e\} = \{ e_1, \ldots, e_d \}$}, set
\mbox{$L := \{ e_1, \ldots, e_i \}$} such that
\[ size(L) \geq \Lfloor \tfrac{1}{2} \Beff \Rfloor \text{ but }
size(L \setminus \{e_i\}) < \Lfloor \tfrac{1}{2} \Beff \Rfloor \]
and \mbox{$R := (E \cup \{e\}) \setminus L$}.
The sizes are bounded as follows:
\begin{align*}
size(L) &\stackrel{\mathrm{def}}{\geq} \Lfloor \tfrac{1}{2} \Beff \Rfloor\\
size(L) &< \Lfloor \tfrac{1}{2} \Beff \Rfloor
		+ \underbrace{size(e_i)}_{\leq \lfloor \frac{1}{n} \Beff \rfloor}
	< \Lfloor \tfrac{n+2}{2n} \Beff \Rfloor\\
size(R) &= \overbrace{size(E \cup \{e\})}^{> \Beff}
		- \overbrace{size(L)}^{< \lfloor \frac{n+2}{2n} \Beff \rfloor}
	> \Lceil \tfrac{n-2}{2n} \Beff \Rceil\\
size(R) &= \underbrace{size(E \cup \{e\})}_{
			\leq \Beff + \lfloor \frac{1}{n} \Beff \rfloor}
		- \underbrace{size(L)}_{\geq \lfloor \frac{1}{2} \Beff \rfloor}
	\leq \Lceil \tfrac{1}{2} \Beff \Rceil + \Lfloor \tfrac{1}{n} \Beff \Rfloor
	\leq \Lceil \tfrac{n+2}{2n} \Beff \Rceil
\end{align*}
By this, it directly follows that 
\mbox{$\Lfloor \tfrac{n-2}{2n} \Beff \Rfloor \leq size(L), size(R) \leq \Beff$},
i.e.  $L$ and $R$ fulfill req.~\ref{minsize}.
\end{proof}
\end{thm}


\begin{thm}[Node-Underflow]
When an entry $e$ is removed from a non-root-node with entry set $L$ such that 
$size(L) \geq \Lfloor \tfrac{n-2}{2n} \Beff \Rfloor$ but
$size(L \setminus \{e\}) < \Lfloor \tfrac{n-2}{2n} \Beff \Rfloor$, either
entries from a neighbor can be moved or $L$ can be merged with a neighbor.

\begin{proof}
Let $R$ be the entry set of a neighbor of $L$.
Depending on $size(R)$, either entries must be moved or nodes must be merged:
\begin{enumerate}
\item
If $\Lfloor \tfrac{n-2}{2n} \Beff \Rfloor \leq size(R) 
\leq \Lfloor \tfrac{n+2}{2n} \Beff \Rfloor$, the nodes $L$ and $R$ can be
merged to a node \mbox{$N := (L \setminus \{e\}) \cup R$}.
The following equation shows that $N$ fulfills req.~\ref{minsize}:
\[ \Lfloor \tfrac{n-2}{2n} \Beff \Rfloor 
\leq size(N)
= \underbrace{size(L \setminus \{e\})}_{< \lfloor \frac{n-2}{2n} \Beff \rfloor}
	+ \underbrace{size(R)}_{\leq \lfloor \frac{n+2}{2n} \Beff \rfloor}
\leq \Beff.\]

\item
If \mbox{$size(R) > \Lfloor \tfrac{n+2}{2n} \Beff \Rfloor$}, one or more entries
can be moved from $R$ to $L$.
We know that
\[ size(L \setminus \{e\})
= \underbrace{size(L)}_{\geq \lfloor \frac{n-2}{2n} \Beff \rfloor}
	- \underbrace{size(e)}_{\leq \lfloor \frac{1}{n} \Beff \rfloor}
\geq \Lfloor \tfrac{n-4}{2n} \Beff \Rfloor. \]
Entries with a cumulated size \mbox{$\geq size(e)$} must be moved from 
$R$ to $L$.
If \mbox{$R = \{e_1, \ldots, e_n\}$}, the set \mbox{$S := \{e_1, \ldots, e_i\}$}
such that
\[ size(S) \geq size(e) \text{ but } size(S \setminus \{e_i\}) < size(e) \]
is the minimum set that must be moved from $R$ to $L$ to make $L$ valid.
In the worst case, \mbox{$S \setminus \{e_i\}$} is verly large but not large
enough, i.e.  \mbox{$size(S \setminus \{e_i\}) + \varepsilon = size(e)$}, and
the last entry $e_i$ is unnecessarily large
\mbox{$size(e_i) = \Lfloor \tfrac{1}{n} \Beff \Rfloor$}.
Hence, the size of $S$ is bounded by
\[ size(S) < 2 \cdot \Lfloor \tfrac{1}{n} \Beff \Rfloor
\leq \Lfloor \tfrac{2}{n} \Beff \Rfloor. \]

Set \mbox{$L' := (L \setminus \{e\}) \cup S$} and \mbox{$R' := R \setminus S$}.
Then the sizes are bounded as follows:
\begin{align*}
size(L') &= \overbrace{size(L \setminus \{e\})}^{
			< \lfloor \frac{n-2}{2n} \Beff \rfloor}
		+ \overbrace{size(S)}^{< \lfloor \frac{2}{2n} \Beff \rfloor}
	< \Lfloor \tfrac{n+2}{2n} \Beff \Rfloor\\
size(L') &= size(L \setminus \{e\}) + size(S)
	\stackrel{\mathrm{def}}{\geq} \Lfloor \tfrac{n-2}{2n} \Beff \Rfloor\\
size(R') &< size(R) \leq \Beff\\
size(R') &= \underbrace{size(R)}_{> \lfloor \frac{n+2}{2n} \Beff \rfloor}
		- \underbrace{size(S)}_{\leq \lfloor \frac{n-4}{2n} \Beff \rfloor}
	\geq \Lfloor \tfrac{n-2}{2n} \Beff \Rfloor
\end{align*}
By this, it directly follows that $L'$ and $R'$ fulfill req.~\ref{minsize}.
\end{enumerate}
\end{proof}
\end{thm}



\section{Algorithms} %%%%%%%%%%%%%%%%%%%%%%%%%%%%%%%%%%%%%%%%%%%%%%%%%%%%%%%%%%

\begin{proc}[Insertion] \label{ins}
To insert a key-value-pair \mbox{$(k, v)$}, in the first step the corresponding
leaf node is searched (proc~\ref{sea}).
Then the node increase handling (proc.~\ref{inc}) is performed.
\end{proc}


\begin{proc}[Deletion] \label{del}
To delete a key $k$, in the first step the corresponding
leaf node is searched (proc~\ref{sea}).
If such a leaf is found, the key-value-pair is removed and then the node
decrease handling (proc.~\ref{dec}) is performed.
\end{proc}


\begin{proc}[Search] \label{sea}
The search for a key $k$ can be carried out by performing the following steps:
\begin{enumerate}
\item Set \mbox{$N := (M, E)$} to the root node with
	\mbox{$M = \{e_1, \ldots, e_d\}$}.
\item \label{sea:loop} Choose \mbox{$i := \arg \max_{j} k \leq key(e_j)$}.
\item If no such $n$ exists
	\begin{itemize}
	\item and the best future position for $k$ is searched,
		set \mbox{$i := deg(N)$} if $N$ is an inner node and
		set \mbox{$i := deg(N) + 1$} if $N$ is a leaf,
	\item and the value for $k$ is searched, exit with an error.
	\end{itemize}
\item If $N$ is a leaf, return $value(e_i)$.
\item Set \mbox{$N := child(e_i)$} and go to step~\ref{sea:loop}.
\end{enumerate}
\end{proc}


\begin{proc}[Node Increase Handling] \label{inc}
Input is a node \mbox{$N = (M, E)$} to which an element was inserted.
Then, the following options are tried; each one is only realized if the 
resulting nodes are valid.
\begin{itemize}
\item[1:] Maybe the node is valid.
\item[2:] Try to shift the \mbox{$\min E$ to $left(N)$}.
\item[3:] Try to shift the \mbox{$\max E$ to $right(N)$}.
\item[4:] Split $N$ into two.
\end{itemize}
This order chosen to minimize the creation of new nodes (in the first place and
disk accesses in the second place).

These steps are in detail (the synchronization and redistributions are 
described below):
\begin{itemize}
\item[ad 1:] Write $N$ to its (old) place and synchronize with parent
	(proc.~\ref{sync}).
\item[ad 2:] Redistribute $left(N)$ and $N$ (proc.~\ref{dist},
	redistribution cares about synchronization).
\item[ad 3:] Redistribute $N$ and $right(N)$ (proc.~\ref{dist},
	redistribution cares about synchronization).
\item[ad 4:] Split $N$ to two nodes $L$ and $R$ (proc.~\ref{split}).
	$R$ replaces $N$, while $L$ is a new node.
	If $N$ was root, create a new root node with the entries $L$ and $R$ and
	set their parents appropriately (this is the only moment when the
	{\em tree grows in height}).
	Otherwise, insert $L$ into its parent and synchronize $R$ with its parent
	(proc.~\ref{ins} and proc.~\ref{sync}; insertion of $L$ cares about its
	synchronization, note that the insertion of $L$ into $parent(L)$
	{\em might have changed} $parent(R)$).
\end{itemize}
\end{proc}


\begin{proc}[Node Decrease Handling] \label{dec}
Input is a node \mbox{$N = (M, E)$} from which an element was deleted.
Then, the following options are tried; each one is only realized if the
resulting nodes are valid.
\begin{itemize}
\item[1:] Maybe $N$ is root and \mbox{$deg(N) \leq 1$}.
\item[2:] Try to merge $N$ with $left(N)$.
\item[3:] Try to merge $N$ with $right(N)$.
\item[4:] Maybe the node is valid.
\item[5:] Try to shift \mbox{$\min right(N)$} to $N$.
\item[6:] Shift \mbox{$\max left(N)$} to $N$.
\end{itemize}
The order is chosen to maximize the removal of nodes (in the first place and
disk accesses in the second place).

These steps are in detail (the synchronization and redistributions are 
described below):
\begin{itemize}
\item[ad 1:] If \mbox{$deg(N) = 0$}, do nothing ($N$ must also be a leaf in
	this case).
	If \mbox{$deg(N) = 1$}, remove $N$ and make its single child new root
	unless the root is already a leaf (this is the only moment the 
	{\em tree decreases in height}).
\item[ad 2+3:] Let \mbox{$L = left(N)$} and \mbox{$R = N$} respectively
	\mbox{$L = N$} and $R = right(N)$.
	Add the the entries of $L$ to $R$.
	Note that, in general, \mbox{$parent(L) \neq parent(R)$}.
	Synchronize $R$ with $parent(R)$ and delete $L$ from $parent(L)$
	and remove the node $L$ 
	(proc.~\ref{sync} and proc.~\ref{del}; deletion of $L$ cares about its
	synchronization, note that the synchronization of $R$ with $parent(R)$
	{\em might have changed} $parent(L)$).
\item[ad 4:] Write $N$ to its (old) place and synchronize with parent
	(proc.~\ref{sync}).
\item[ad 5:] Redistribute $N$ and $right(N)$ (proc.~\ref{dist},
	redistribution cares about synchronization).
\item[ad 6:] Redistribute $left(N)$ and $N$ (proc.~\ref{dist},
	redistribution cares about synchronization).
\end{itemize}
\end{proc}


\begin{proc}[Optimal Split of One Node] \label{split}
Let \mbox{$E = \{ e_1, \ldots, e_d \}$} be the entries of node
\mbox{$N = (M, E)$} that should be distribed on two nodes 
\mbox{$L = (M_1, E_1)$} and \mbox{$R = (M_2, E_2)$}.
The goal is to calculate
\[ \arg \max_i \underbrace{\min_i \{ size(\{e_1, \ldots, e_i\}),
	size(\{e_{i+1}, \ldots, e_d\}) \}}_{=: s(i)} \]
\begin{enumerate}
\item Set \mbox{$i := 0$}.
\item \label{split:loop} If \mbox{$s(i+1) > s(i)$},  \mbox{$i := i + 1$}. 
\item Otherwise exit with $i$.
\item Go to \ref{split:loop}.
\end{enumerate}
Then set \mbox{$E_1 := \{e_1, \ldots, e_i\}$}
and \mbox{$E_2 := \{e_{i+1}, \ldots, e_d\}$}.
Set \mbox{$parent(L) := parent(R) := parent(N)$}, \mbox{$left(L) := left(N)$}
\mbox{$right(L) := R$}, \mbox{$left(R) := L$}, \mbox{$right(R) := right(N)$}.

\begin{proof}
Initially, $E_1$ contains no elements.
In the following iterations, in step~\ref{split:loop}, the $n + 1$-th element
is moved from $E_2$ to $E_1$ as long as this movement makes the smaller 
of the two nodes larger.
\end{proof}
\end{proc}


\begin{proc}[Node Redistribution] \label{dist}
Let $L$ and $R$ be two nodes. They are combined two a single node $N$
(which is overflown).
This is split optimally to $L'$ and $R'$ (proc.~\ref{split}).
Note that either children move from $L$ to $R'$ or from $R$ to $L'$.
Synchronize $L$ and $R$ with $parent(L)$ and $parent(R)$ (proc.~\ref{sync};
note that the synchronization of $L$ with $parent(L)$ {\em might have changed}
$parent(R)$).
\end{proc}


\begin{proc}[Node Synchronization] \label{sync}
Let \mbox{$N = (M_N, E_N)$} be a node and \mbox{$P = (M_P, E_P)  = parent(N)$}
its parent.
Then let \mbox{$e_i \in E_P$} with \mbox{$child(e_i) = N$} be the entry of
$N$ in $P$.
The main task of the synchronization is to
\begin{itemize}
\item set \mbox{$key(e_i) := \max N$}, and
\item set \mbox{$count(e_i) := \sum_{e \in E_N} count(e)$}.
\end{itemize}
Let $e_i'$ be the entry $e_i$ before its modifications.

The modifications might have the effect that the $P$ is {\em not valid anymore}:
\begin{itemize}
\item If \mbox{$size(\max N) < size(key(e_i'))$},
	$P$ might violate req.~\ref{minsize} after the update.
	A node decrease handling (proc.~\ref{dec}) of $P$ must be performed.
\item If \mbox{$size(\max N) > size(key(e_i'))$},
	$P$ might violate req.~\ref{maxsize} after the update.
	A node increase handling (proc.~\ref{inc}) of $P$ must be performed.
\item Otherwise, the $P$ must be written back and, if $i = deg(N)$ (and
	$key(e_i) \neq key(e_i')$) or if $count(e_i) \neq count(e_i')$, $P$
	must be synchronized with its parent (proc.~\ref{sync}).
\end{itemize}

The increase and decrease handling procedures care about writing back $P$
and further synchronizations.

Note that the increase and decrease handlings might have effect on the
neighbors of $P$, and also on their children and parents.
This means that a synchronization of a node $L$ might have the effect that
the parent $parent(R)$ of some sibling $R$ of $L$ changes.
\end{proc}



\section{Summary}

One disadvantage of storing the indexes of the elements to offer access by
the index number is that each insertion and each deletion needs to write
the nodes from the root to the leaf.
This raises problems with locking, because the root node becomes to a
bottleneck that has to be locked for each operation.
I do not know any good solution for this problem.

The Dingsbums 6 database system uses an Ada 95 implementation of this
\BTree as its core data structure.
The problem of the root node being a bottleneck is not that important in this
context since Dingsbums 6 is intended to be a read-optimized database system.

The definitions, proofs and procedures are hopefully correct.
The \BTree has tested with one billion entries, but there is still potential
for errors.



%\nocite{*}
\bibliography{bibliography}
\end{multicols}

\end{document}

