% vim:foldmethod=marker:tabstop=4:shiftwidth=4
\documentclass[a4paper, 8pt]{scrartcl}
\usepackage[english]{babel}
\usepackage[latin1]{inputenc}
\usepackage{amsthm,amsmath,amssymb}
\usepackage{typearea}
\usepackage{multicol}
\usepackage{natbib}
\usepackage{times}


\setlength{\columnsep}{9mm}
%\setlength{\columnseprule}{.1pt}
\areaset{0.8\paperwidth}{0.8\paperheight}

\theoremstyle{plain}

\theoremstyle{definition}
\newtheorem{thm}{Theorem}[section]
\newtheorem{lem}[thm]{Lemma}
\newtheorem{prop}[thm]{Proposition}
\newtheorem{algo}[thm]{Algorithm}
\newtheorem{proc}[thm]{Procedure}
\newtheorem{defi}[thm]{Definition}
\newtheorem{notation}[thm]{Notation}

\theoremstyle{remark}
\newtheorem{rem}[thm]{Remark}

\newcommand \BTree { B\textsuperscript{+}-Tree }
\newcommand \BTrees { B\textsuperscript{+}-Trees }
\newcommand \Beff { B_{\text{eff}} }

\newcommand \sfrac[2] { {}^{#1}\!\!\!\diagup\!\!_{#2} }

% don't hyphenate so much - default = 200, max (never hyphenate) = 10,000
%\hyphenpenalty=10000

\title{Dingsbums \BTree}
\date{October 15, 2008}
\author{Christoph Schwering\\{schwering@gmail.com}}

\begin{document}
\bibliographystyle{plain}


\maketitle

\begin{abstract}
This article gives a formal description of the \BTree used in the Dingsbums 6
database system.
A \BTree is an order-preserving index structure for block oriented storage
devices.
The B-Tree was first described by Bayer and McCreight \cite{Bayer}.
Several variants of the B-Tree appeared in the aftermath, in particular the
\BTree \cite{Cormen, KnuthBTree}, which differs from the B-Tree in that it
carries the values only in its leaves.
Knuth also mentions a variant that is capable of storing variable-length
keys, but does not describe it further \cite{KnuthBTree}.

The \BTree described in this article supports bounded length entries 
and access by the index number of entries (both features are independent from
another, though).
Ignoring some rather small internal overhead and the value sizes for the 
moment, each entry's size is bounded by $\tfrac{1}{4}$ of the node size
in storage elements, and the usage of each node in the tree is at least
$\tfrac{3}{8}$ of its size in storage elements.
\end{abstract}


\begin{multicols}{2}


\section{Definition} %%%%%%%%%%%%%%%%%%%%%%%%%%%%%%%%%%%%%%%%%%%%%%%%%%%%%%%%%%

\begin{defi}[Node]
A node $N$ is a set \mbox{$\{ e_1, \ldots, e_d \}$} of entries. 
The degree of $N$ is the count of entries \mbox{$deg(N) = d$}.

Besides the entries, there is some meta data $meta(N)$ that contains the
degree $deg(N)$, the neighbors' addresses $left(N)$ and $right(N)$ and its
parent's address $parent(N)$.
If such a neighbor or parent nodes does not exist, the respective value is
$\bot$.

The form of an entry depends on the position of the node.
An entry $e_i$ of a {\em inner node} consists of
\begin{itemize}
\item a key value $key(e_i)$,
\item a count $count(e_i)$, and
\item an address $child(e_i)$.
\end{itemize}
An entry $e_i$ of a {\em leaf node} consists of
\begin{itemize}
\item a key value $key(e_i)$, and
\item a data value $value(e_i)$.
\end{itemize}
For inner nodes, we write \mbox{$e = (k, c, a)$} and for leaf nodes 
\mbox{$e = (k, v)$}.
If the context permits it, we do not distinguish between addresses and nodes,
i.e. \mbox{$parent(child(e_i)) = N$}, for example.
If $N$ is a leaf node, $count(e)$ is defined as $1$.
\end{defi}

\begin{defi}[Tree]
A \BTree is defined over a domain $K$ of keys that is fully ordered by
\mbox{$\leq \ \subseteq K \times K$} and a domain $V$ of values.
We write \mbox{$e_i \leq e_j$} in the following iff
\mbox{$key(e_i) \leq key(e_j)$}.
The tree supports the operations {\em insertion}, {\em deletion} and 
{\em search}. We can define these operations by looking at the states of the
tree:
\begin{itemize}
\item In state $S_0$, the tree is empty and for all $k$,
	\mbox{$search(S_0, k) = \bot$} (i.e. no value $v$ for $k$ can be found).
\item In a state $S_i$, the operation \mbox{$insert(S_i, k, v)$} reaches a
	state $S_{i+1}$ in which \mbox{$search(S_{i+1}, k) = v$},
	if \mbox{$search(S_i, k) = \bot$}. Otherwise, the insertion fails and the
	tree state does not change.
\item In a state $S_i$, the operation \mbox{$delete(S_i, k)$} returns $v$ and
	reaches a state $S_{i+1}$ in which \mbox{$search(S_{i+1}, k) = \bot$}.
\end{itemize}
In all states, the key condition which makes searching efficient must hold
for all nodes $N$ in the tree:
\begin{align*}
\forall e \in N: \ & child(e) = N'\\
     \Rightarrow \ & \forall e' \in N': e' \leq e.
\end{align*}
Furthermore, the counts must be consistent in all states:
\begin{align*}
\forall e \in N: \ & child(e) = N'\\
     \Rightarrow \ & count(e) = \sum_{e' \in N'} count(e').
\end{align*}

\begin{figure*}[t]
\newdimen \shortlinewidth \shortlinewidth=0.60\linewidth
\centering
\begin{minipage}{\shortlinewidth}
\begin{flalign}
& size(e) \leq \lfloor \tfrac{1}{4} \Beff \rfloor
	&& \text{ for all } e \in N \label{maxsize}
    &\\
& \lfloor \tfrac{3}{8} \Beff \rfloor \leq size(N) \leq \Beff
	&& \text{ for all } N \text{ except the root} \label{minsize}
    &\\
& 0 \leq size(R) \leq \Beff
	&& \text{ for the root node } R \label{rootminsize}
    &
\end{flalign}
\caption{Node filling requirements}
\hspace{2cm}
\hrule
\end{minipage}
\label{fill}
\end{figure*}


\noindent {\em Size Limits.}
Normal \BTrees require their nodes to have a certain fixed minimum and maximum
degree, because each node is intended to be stored in one hard disk block of
size $B$ (typically $4096$ bytes).
In contrast, our tree requires each node's storage space to be used to a
minimum.
This enables the tree to support {\em keys and values of variable length}.

Note that for all nodes $N$, $size_\text{meta} = size(meta(N))$ is constant,
and for all count values $c$ and child addresses $a$, $size_\text{count} =
size(c)$ and $size_\text{addr} = size(a)$ are constant.

We write $\Beff = B - size_\text{meta}$ for the effective node size and leave
out the size of meta data in $size(N) = \sum_{e \in N} size(e)$ because we
always compare it to $\Beff$.
For each $e \in N$, the size is determined by
\[ size(e) = \begin{cases}
size(k) + size_\text{count} + size_\text{addr} & \text{if } e = (k,c,a)\\
size(k) + size(v)                              & \text{if } e = (k,v)
\end{cases}. \]

Finally, we require for each node $N$ two hold the requirements depicted in
figure~\ref{fill} on page~\pageref{fill}.
\end{defi}


\begin{rem}[Key size]
Note that req.~\ref{maxsize} refers to a node element $e$ which consists
of either \mbox{$e = (k, c, a)$} or \mbox{$e = (k, v)$}.
It follows that the maximum key size is
\begin{align*}
size(k) + size_\text{count} + size_\text{addr}
    &\leq \lfloor \tfrac{1}{4} \Beff \rfloor & \text{ and}\\
size(k) + size(v)
    &\leq \lfloor \tfrac{1}{4} \Beff \rfloor.
\end{align*}
Hence, if the size $size(v)$ of all data values $v$ is bounded by
$size_\text{value}$, the maximum key size is bounded as follows:
\[ size(k) \leq \lfloor \tfrac{1}{4} \Beff \rfloor 
    - \max \{ size_\text{count} + size_\text{addr}, size_\text{value} \}. \]
\end{rem}


\begin{rem}[Degeneration]
The tree cannot perform worse than a binary tree, because even in the worst
case, each node except for the root node must hold at least two elements.
This is because the minimal node usage is greater than the maximum entry size.
\end{rem}



\section{Theorems} %%%%%%%%%%%%%%%%%%%%%%%%%%%%%%%%%%%%%%%%%%%%%%%%%%%%%%%%%%%%

\begin{thm}[Node-Overflow]
A node with entry set \mbox{$N \cup \{ e \}$} with
\[ size(N) \leq \Beff \text{ but } size(N \cup \{e\}) > \Beff \]
can be split into two nodes such that both fulfill req.~\ref{minsize}.

\begin{proof}
For \mbox{$N \cup \{e\} = \{ e_1, \ldots, e_n \}$}, set
\mbox{$L := \{ e_1, \ldots, e_i \}$} such that
\[ size(L) \geq \lfloor \tfrac{3}{8} \Beff \rfloor \text{ but }
	size(L \setminus \{e_i\}) < \lfloor \tfrac{3}{8} \Beff \rfloor \]
and \mbox{$R := (N \cup \{e\}) \setminus L$}.
Then the sizes are bounded as follows:
\begin{align*}
size(L) &\stackrel{\mathrm{def}}{\geq} \lfloor \tfrac{3}{8} \Beff \rfloor\\
size(L) &< \lfloor \tfrac{3}{8} \Beff \rfloor
		+ \underbrace{size(e_i)}_{\leq \lfloor \frac{1}{4} \Beff \rfloor}
	\leq \lfloor \tfrac{5}{8} \Beff \rfloor\\
size(R) &= \overbrace{size(N \cup \{e\})}^{> \Beff}
		- \overbrace{size(L)}^{< \lfloor \frac{5}{8} \Beff \rfloor}
	> \lceil \tfrac{3}{8} \Beff \rceil\\
size(R) &= \underbrace{size(N \cup \{e\})}_{
			\leq \Beff + \lfloor \frac{1}{4} \Beff \rfloor}
		- \underbrace{size(L)}_{\geq \lfloor \frac{3}{8} \Beff \rfloor}
	\leq \lceil \tfrac{7}{8} \Beff \rceil
\end{align*}
By this, it directly follows that 
\mbox{$\lfloor \tfrac{3}{8} \Beff \rfloor \leq size(L), size(R) \leq \Beff$}, 
i.e.  $L$ and $R$ fulfill req.~\ref{minsize}.
\end{proof}
\end{thm}


\begin{thm}[Node-Underflow]
When an entry $e$ is removed from a non-root-node with entry set $L$ such that 
$size(L) \geq \lfloor \tfrac{3}{8} \Beff \rfloor$ but
$size(L \setminus \{e\}) < \lfloor \tfrac{3}{8} \Beff \rfloor$, either entries
from a neighbor can be moved or $L$ can be merged with a neighbor.

\begin{proof}
Let \[\delta := \lfloor \tfrac{3}{8} \Beff \rfloor - size(L \setminus \{e\})\]
be the space in \mbox{$L \setminus \{e\}$} that needs to be filled so that
\mbox{$L \setminus \{e\}$} fulfills req.~\ref{minsize}.
Without loss of generality, we assume that the right neighbor with entry set $R$
of $L$ exists; otherwise, a left neighbor exists and instead of the neighbor's
minimal elements its maximal elements must be moved to $L$.

Depending on $size(R)$, either entries must be moved or nodes must be merged:
\begin{enumerate}
\item If \mbox{$size(R) \leq \lceil \frac{5}{8} \Beff \rceil + \delta$},
the nodes can be merged to a single node
\mbox{$N := (L \setminus \{e\}) \cup R$}:
\begin{align*}
size(N)
	&= \overbrace{size(L \setminus \{e\})}^{
			\geq \lfloor \frac{3}{8} \Beff \rfloor 
				- \lfloor \frac{1}{4} \Beff \rfloor}
		+ \overbrace{size(R)}^{
			\geq \lfloor \frac{3}{8} \Beff \rfloor}
	\geq \lfloor \tfrac{1}{2} \Beff \rfloor\\
size(N)
	&= \underbrace{size(L \setminus \{e\})}_{
			\leq \lfloor \frac{3}{8} \Beff \rfloor - \delta}
		+ \underbrace{size(R)}_{
			\leq \lceil \frac{5}{8} \Beff \rceil + \delta}
	\leq \Beff
\end{align*}

\item
If \mbox{$size(R) > \lceil \frac{5}{8} \Beff \rceil + \delta$}, a set
\mbox{$S \subseteq R$} must be moved from $R$ to $L$.
We then set \mbox{$L' := (L \setminus \{e\}) \cup S$} and
\mbox{$R' := R \setminus S$}.
The set $S$ must be minimal, i.e.
\[ size(L') \geq \lfloor \tfrac{3}{8} \Beff \rfloor \text{ but }
	size(L' \setminus \max S) < \lfloor \tfrac{3}{8} \Beff \rfloor. \]
It follows that
\[ \delta \leq size(S) < \delta + \lfloor \tfrac{1}{4} \Beff \rfloor. \]
Hence, $L'$ and $R'$ meet req.~\ref{minsize}:
\begin{align*}
size(L')
	&\stackrel{\mathrm{def}}{\geq} \lfloor \tfrac{3}{8} \Beff \rfloor\\
size(L')
	&= \underbrace{size(L \setminus \{e\})}_{
			= \lfloor \frac{3}{8} \Beff \rfloor - \delta}
		+ \underbrace{size(S)}_{
			< \delta + \lfloor \frac{1}{4} \Beff \rfloor}
	< \lfloor \tfrac{5}{8} \Beff \rfloor\\
size(R')
	&= \overbrace{size(R)}^{
			> \lceil \frac{5}{8} \Beff \rceil + \delta}
		- \overbrace{size(S)}^{
			< \delta + \lfloor \frac{1}{4} \Beff \rfloor}
	> \lfloor \tfrac{3}{8} \Beff \rfloor\\
size(R')
	&< size(R) \leq \Beff
\end{align*}
\end{enumerate}
\end{proof}
\end{thm}



\section{Algorithms} %%%%%%%%%%%%%%%%%%%%%%%%%%%%%%%%%%%%%%%%%%%%%%%%%%%%%%%%%%

\begin{proc}[Insertion] \label{ins}
To insert a key-value-pair \mbox{$(k, v)$}, in the first step the corresponding
leaf node is searched (proc~\ref{sea}).
Then the node increase handling (proc.~\ref{inc}) is performed.
\end{proc}


\begin{proc}[Deletion] \label{del}
To delete a key $k$, in the first step the corresponding
leaf node is searched (proc~\ref{sea}).
If such a leaf is found, the key-value-pair is removed and then the node
decrease handling (proc.~\ref{dec}) is performed.
\end{proc}


\begin{proc}[Search] \label{sea}
The search for a key $k$ can be carried out by performing the following steps:
\begin{enumerate}
\item Set $N$ to the root node with \mbox{$N = \{e_1, \ldots, e_d\}$}.
\item \label{sea:loop} Choose \mbox{$n := \arg \max_{i} k \leq key(e_i)$}.
\item If no such $n$ exists
	\begin{itemize}
	\item and the best future position for $k$ is searched,
		set \mbox{$n := deg(N)$} if $N$ is an inner node and
		set \mbox{$n := deg(N) + 1$} if $N$ is a leaf,
	\item and the value for $k$ is searched, exit with an error.
	\end{itemize}
\item If $N$ is a leaf, return $value(e_n)$.
\item Set \mbox{$N := child(e_n)$} and go to step~\ref{sea:loop}.
\end{enumerate}
\end{proc}


\begin{proc}[Node Increase Handling] \label{inc}
Input is a node $N$ to which an element was inserted.
Then, the following options are tried; each one is only realized if the 
resulting nodes are valid.
\begin{enumerate}
\item Maybe the node is valid.
\item Try to shift the \mbox{$\min N$ to $left(N)$}.
\item Try to shift the \mbox{$\max N$ to $right(N)$}.
\item Split $N$ into two.
\end{enumerate}
This order chosen to minimize the creation of new nodes (in the first place and
disk accesses in the second place).

These steps are in detail (the synchronization and redistributions are 
described below):
\begin{enumerate}
\item[ad 1.] Write $N$ to its (old) place and synchronize with parent
	(proc.~\ref{sync}).
\item[ad 2.] Redistribute $left(N)$ and $N$ (proc.~\ref{dist},
	redistribution cares about synchronization).
\item[ad 3.] Redistribute $N$ and $right(N)$ (proc.~\ref{dist},
	redistribution cares about synchronization).
\item[ad 4.] Split $N$ to two nodes $L$ and $R$ (proc.~\ref{split}).
	$R$ replaces $N$, while $L$ is a new node.
	If $N$ was root, create a new root node with the entries $L$ and $R$ and
	set their parents appropriately (this is the only moment when the
	{\em tree grows in height}).
	Otherwise, insert $L$ into its parent and synchronize $R$ with its parent
	(proc.~\ref{ins} and proc.~\ref{sync}; insertion of $L$ cares about its
	synchronization, note that the insertion of $L$ into $parent(L)$
	{\em might have changed} $parent(R)$).
\end{enumerate}
\end{proc}


\begin{proc}[Node Decrease Handling] \label{dec}
Input is a node $N$ from which an element was deleted.
Then, the following options are tried; each one is only realized if the
resulting nodes are valid.
\begin{enumerate}
\item Maybe $N$ is root and \mbox{$deg(N) \leq 1$}.
\item Try to merge $N$ with $left(N)$.
\item Try to merge $N$ with $right(N)$.
\item Maybe the node is valid.
\item Try to shift \mbox{$\min right(N)$} to $N$.
\item Shift \mbox{$\max left(N)$} to $N$.
\end{enumerate}
The order is chosen to maximize the removal of nodes (in the first place and
disk accesses in the second place).

These steps are in detail (the synchronization and redistributions are 
described below):
\begin{itemize}
\item[ad 1:] If \mbox{$deg(N) = 0$}, do nothing ($N$ must also be a leaf in
	this case).
	If \mbox{$deg(N) = 1$}, remove $N$ and make its single child new root
	unless the root is already a leaf (this is the only moment the 
	{\em tree decreases in height}).
\item[ad 2+3:] Let \mbox{$L = left(N)$} and \mbox{$R = N$} respectively
	\mbox{$L = N$} and $R = right(N)$.
	Add the the entries of $L$ to $R$.
	Note that, in general, \mbox{$parent(L) \neq parent(R)$}.
	Synchronize $R$ with $parent(R)$ and delete $L$ from $parent(L)$
	and remove the node $L$ 
	(proc.~\ref{sync} and proc.~\ref{del}; deletion of $L$ cares about its
	synchronization, note that the synchronization of $R$ with $parent(R)$
	{\em might have changed} $parent(L)$).
\item[ad 4:] Write $N$ to its (old) place and synchronize with parent
	(proc.~\ref{sync}).
\item[ad 5:] Redistribute $N$ and $right(N)$ (proc.~\ref{dist},
	redistribution cares about synchronization).
\item[ad 6:] Redistribute $left(N)$ and $N$ (proc.~\ref{dist},
	redistribution cares about synchronization).
\end{itemize}
\end{proc}


\begin{proc}[Optimal Split of One Node] \label{split}
Let \mbox{$N = \{ e_1, \ldots, e_d \}$} be the node that should be split to
two nodes $L$ and $R$.
The goal is to calculate
\[ \arg \max_n \underbrace{\min_n \{ size(\{e_1, \ldots, e_{n-1}\}),
	size(\{e_n, \ldots, e_d\}) \}}_{=: d(n)}. \]
\begin{enumerate}
\item Set \mbox{$n := 0$}.
\item \label{split:loop} If \mbox{$d(n) > d(n+1)$}, exit with $n$.
\item Otherwise set \mbox{$n := n + 1$}. 
\item Go to \ref{split:loop}.
\end{enumerate}
Then set \mbox{$L := \{e_1, \ldots, e_{n-1}\}$}
and \mbox{$R := \{e_n, \ldots, e_d\}$}.
Set \mbox{$parent(L) := parent(R) := parent(N)$}, \mbox{$left(L) := left(N)$},
\mbox{$right(L) := R$}, \mbox{$left(R) := L$}, \mbox{$right(R) := right(N)$}.

\begin{proof}
For the initial setting $n = 1$, $L$ contains no elements.
In the $n$-th iteration, the $n$-th element is moved from $R$ to $L$.
The procedure stops in the $n$-th iteration if the $n+1$-th iteration would
worsen the situation, i.e. would make the smaller node even smaller.
As $L$ grows with each iteration, $R$ would be the smaller node in the
$n+1$-th iteration. In any further iteration, $R$ would become even smaller.
It follows that no further iteration would improve the situation.
\end{proof}
\end{proc}


\begin{proc}[Node Redistribution] \label{dist}
Let $L$ and $R$ be two nodes. They are combined two a single node $N$
(which is overflown).
This is split optimally to $L'$ and $R'$ (proc.~\ref{split}).
Note that either children move from $L$ to $R'$ or from $R$ to $L'$.
Synchronize $L$ and $R$ with $parent(L)$ and $parent(R)$ (proc.~\ref{sync};
note that the synchronization of $L$ with $parent(L)$ {\em might have changed}
$parent(R)$).
\end{proc}


\begin{proc}[Node Synchronization] \label{sync}
Let $N$ be a node and $P = parent(N)$ its parent.
Then let \mbox{$e_i \in P$} with \mbox{$child(e_i) = N$} be the entry of
$N$ in $P$.
The main task of the synchronization is to
\begin{itemize}
\item set \mbox{$key(e_i) := key(\max N)$}, and
\item set \mbox{$count(e_i) := \sum_{e \in N} count(e)$}.
\end{itemize}
Let $e_i'$ be the entry $e_i$ before its modifications.

The modifications might have the effect that the $P$ is {\em not valid anymore}:
\begin{itemize}
\item If \mbox{$size(key(\max N)) < size(key(e_i'))$},
	$P$ might violate req.~\ref{minsize} after the update.
	A node decrease handling (proc.~\ref{dec}) of $P$ must be performed.
\item If \mbox{$size(key(\max N)) > size(key(e_i'))$},
	$P$ might violate req.~\ref{maxsize} after the update.
	A node increase handling (proc.~\ref{inc}) of $P$ must be performed.
\item Otherwise, the $P$ must be written back and, if $i = deg(N)$ (and
	$key(e_i) \neq key(e_i')$) or if $count(e_i) \neq count(e_i')$, $P$
	must be synchronized with its parent (proc.~\ref{sync}).
\end{itemize}

The increase and decrease handling procedures care about writing back $P$
and further synchronizations.

Note that the increase and decrease handlings might have effect on the
neighbors of $P$, and also on their children and parents.
This means that a synchronization of a node $L$ might have the effect that
the parent $parent(R)$ of some sibling $R$ of $L$ changes.
\end{proc}



\section{Summary}

One disadvantage of storing the indexes of the elements to offer access by
the index number is that each insertion and each deletion needs to write
all nodes along a path from the root to the leaf.
The reorganization operations due to variable length keys even increase the 
number of nodes visited on the way from the leaf node back to the root.

This not only leads to many IO operations and thereby slows down insertion and
deletion; it also raises problems with concurrency, because the root node
becomes to a bottleneck that has to be locked for each operation.

Currently the concurrency issue is dealt with as follows: during an insertion or
deletion other writers are locked out, but readers are still allowed. The
operations have effect only on copies of the nodes in main memory. Before 
writing them back to disk, also concurrent readers are locked out.

Another way to deal with concurrency would be to slightly modify the \BTree so
that leaf nodes have no lower bound in size and apply locking as proposed in
\cite{Lehman}.

Obviously, the described \BTree is only suitable for cases in which much more
searches than insertion or deletions are performed.
The Dingsbums 6 database system uses an Ada 95 implementation of this
\BTree as its core data structure.

%\nocite{*}
\bibliography{bibliography}
\end{multicols}

\end{document}

